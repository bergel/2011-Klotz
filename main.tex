
%\documentclass{sig-alternate}

% English flaws:
% "can be made ON ..." and not "can be made upon"
% "disadvantage OF disruptING" and not " disadvantage to disrupt"
% "requesting something" and not "requesting for something"


%\documentclass[runningheads]{llncs}

%\documentclass[times, 10pt,twocolumn]{article} 
%\documentclass[preprint,10pt]{sigplanconf}

\documentclass[runningheads]{llncs}
%\documentclass{sig-alternate}

% constants
\newcommand{\Title}{Klotz: Agile 3D Visualization}
\newcommand{\TitleShort}{\Title}
\newcommand{\Authors}{Ricardo Jacas~~~~~Alexandre Bergel}
\newcommand{\AuthorsShort}{R. Jacas, A. Bergel}

% packages
\usepackage{xspace}
\usepackage{ifthen}
\usepackage{amsbsy}
\usepackage{amssymb}
\usepackage{balance}
\usepackage{booktabs}
\usepackage{graphicx}
\usepackage{multirow}
\usepackage{needspace}
\usepackage{microtype}
\usepackage{bold-extra}


% references
\usepackage[colorlinks]{hyperref}
\usepackage[all]{hypcap}
\setcounter{tocdepth}{2}
\hypersetup{
	colorlinks=true,
	urlcolor=black,
	linkcolor=black,
	citecolor=black,
	plainpages=false,
	bookmarksopen=true,
	pdfauthor={\Authors},
	pdftitle={\Title}}

\def\chapterautorefname{Chapter}
\def\appendixautorefname{Appendix}
\def\sectionautorefname{Section}
\def\subsectionautorefname{Section}
\def\figureautorefname{Figure}
\def\tableautorefname{Table}
\def\listingautorefname{Listing}

% source code
\usepackage{xcolor}
\usepackage{textcomp}
\usepackage{listings}
\definecolor{source}{gray}{0.9}
\lstset{
	language={},
	% characters
	tabsize=3,
	upquote=true,
	escapechar={!},
	keepspaces=true,
	breaklines=true,
	alsoletter={\#:},
	breakautoindent=true,
	columns=fullflexible,
	showstringspaces=false,
	basicstyle=\footnotesize\sffamily,
	% background
	frame=single,
    framerule=0pt,
	backgroundcolor=\color{source},
	% numbering
	numbersep=5pt,
	numberstyle=\tiny,
	numberfirstline=true,
	% captioning
	captionpos=b,
	% formatting (html)
	moredelim=[is][\textbf]{<b>}{</b>},
	moredelim=[is][\textit]{<i>}{</i>},
	moredelim=[is][\color{red}\uwave]{<u>}{</u>},
	moredelim=[is][\color{red}\sout]{<del>}{</del>},
	moredelim=[is][\color{blue}\underline]{<ins>}{</ins>}}
\newcommand{\ct}{\lstinline[backgroundcolor=\color{white},basicstyle=\footnotesize\ttfamily]}
\newcommand{\lct}[1]{{\small\tt #1}}

% tikz
% \usepackage{tikz}
% \usetikzlibrary{matrix}
% \usetikzlibrary{arrows}
% \usetikzlibrary{external}
% \usetikzlibrary{positioning}
% \usetikzlibrary{shapes.multipart}
% 
% \tikzset{
% 	every picture/.style={semithick},
% 	every text node part/.style={align=center}}

% proof-reading
\usepackage{xcolor}
\usepackage[normalem]{ulem}
\newcommand{\ra}{$\rightarrow$}
\newcommand{\ugh}[1]{\textcolor{red}{\uwave{#1}}} % please rephrase
\newcommand{\ins}[1]{\textcolor{blue}{\uline{#1}}} % please insert
\newcommand{\del}[1]{\textcolor{red}{\sout{#1}}} % please delete
\newcommand{\chg}[2]{\textcolor{red}{\sout{#1}}{\ra}\textcolor{blue}{\uline{#2}}} % please change
\newcommand{\chk}[1]{\textcolor{ForestGreen}{#1}} % changed, please check

% comments \nb{label}{color}{text}
\newboolean{showcomments}
\setboolean{showcomments}{true}
\ifthenelse{\boolean{showcomments}}
	{\newcommand{\nb}[3]{
		{\colorbox{#2}{\bfseries\sffamily\scriptsize\textcolor{white}{#1}}}
		{\textcolor{#2}{\sf\small$\blacktriangleright$\textit{#3}$\blacktriangleleft$}}}
	 \newcommand{\version}{\emph{\scriptsize$-$Id$-$}}}
	{\newcommand{\nb}[2]{}
	 \newcommand{\version}{}}
\newcommand{\rev}[2]{\nb{Reviewer #1}{red}{#2}}
\newcommand{\ab}[1]{\nb{Alexandre}{blue}{#1}}
\newcommand{\vp}[1]{\nb{Vanessa}{orange}{#1}}

% graphics: \fig{position}{percentage-width}{filename}{caption}
\DeclareGraphicsExtensions{.png,.jpg,.pdf,.eps,.gif}
\graphicspath{{figures/}}
\newcommand{\fig}[4]{
	\begin{figure}[#1]
		\centering
		\includegraphics[width=#2\textwidth]{#3}
		\caption{\label{fig:#3}#4}
	\end{figure}}
\newcommand{\largefig}[4]{
	\begin{figure*}[#1]
		\centering
		\includegraphics[width=#2\textwidth]{#3}
		\caption{\label{fig:#3}#4}
	\end{figure*}}

% abbreviations
\newcommand{\ie}{\emph{i.e.,}\xspace}
\newcommand{\eg}{\emph{e.g.,}\xspace}
\newcommand{\etc}{\emph{etc.}\xspace}
\newcommand{\etal}{\emph{et al.}\xspace}

% lists
\newenvironment{bullets}[0]
	{\begin{itemize}}
	{\end{itemize}}

\newcommand{\seclabel}[1]{\label{sec:#1}}
\newcommand{\figlabel}[1]{\label{fig:#1}}
\newcommand{\figref}[1]{Figure~\ref{fig:#1}}
\newcommand{\secref}[1]{Section~\ref{sec:#1}}
\newcommand{\tablabel}[1]{\label{tab:#1}}
\newcommand{\tabref}[1]{Table~\ref{tab:#1}}

\newcommand{\myparagraph}[1]{\noindent \textbf{#1.}}

%%%%%%%%%%%%%%%%%%%%%%%%%

\begin{document}
%SPRINGER
%\title{Reconciling Static Type Declaration and Dynamic Scripting Languages}
%\author{Alexandre Bergel}
%\authorrunning{A. Bergel}
%\institute{ADAM Project, INRIA Futurs, Lille, France\\
%\co{\href{http://www.bergel.eu}{www.bergel.eu}}\\
%}

%ACM
%\conferenceinfo{ECOOP}{'11}

\title{\Title}
\titlerunning{\Title}

%\author{Alexandre Bergel\\
%Pleiad Lab, Department of Computer Science (DCC),\\ University of Chile, Santiago, Chile\\ [1 ex]
%\href{http://bergel.eu}{http://bergel.eu}
%}

\author{\Authors}
\authorrunning{\AuthorsShort}

\institute{PLEIAD Lab, Department of Computer Science (DCC), \\University of Chile, Santiago, Chile\\
\url{ricardo.jacas@gmail.com}~~~~
\url{http://bergel.eu}
}


\newcommand{\spp}{~~~~~~~}

\maketitle

%\cvsversion

%\begin{center}
%\textbf{Accepted at ECOOP'11 - Do not distribute, this is not the final version}
%\end{center}

\begin{abstract}
% What is the problem


% Why is the problem a problem?
% What's the surprising Idea?
% What's the consequence?
\end{abstract}

%\category{D.3.3}{Programming Languages}{Language Constructs and Features}
%\category{D.1.5}{Programming Languages}{Object-oriented Programming}
%\terms{Language, Design}
%\keywords{Multi-language system, interoperability, SmalltalkLite, JavaLite, dynamic languages}


%\begin{figure}[!]
%\begin{center}
%\includegraphics[scale=0.68]{figures/controlleddisruption}
%\caption{The method \co{senseAndSend} is cut down into small pieces, called fragments.} \figlabel{controlleddisruption}
%\end{center}
%\end{figure}

%:%%%%%%%%%%%%%%%%%%%%%%%%%%%%%%%%%%%%%%%%%%%%%%%%%%
%\emph{Note for the proceeding reader: this paper makes use of colors. Although not mandatory for its understanding, an online (colored) version of this paper will ease the reading.}

\section{Introduction} \seclabel{introduction}

% What is the problem
% Why is the problem a problem?
% What's the surprising Idea?
% What's the consequence?

%:%%%%%%%%%%%%%%%%%%%%%%%%%%%%%%%%%%%%%%%%%%%%%%%%%%
\section{Klotz} \seclabel{klotz}

%=======
\subsection{Klotz in a nutshell} \seclabel{nutshell}
Klotz is a \chg{F}{f}ramework, based on simple 3D graphic elements,
that are meant to work as metrics to specify characteristics of
code. The elements to work with are cubes and edges between them.

The goal of Klotz is to offer a flexible solution to visualize software 
in 3 dimensions. Klotz offer\ins{s} a scripting language, an API and an easel to 
``compose'' a visualization. 

The example below shows this very main goal: the script creates a group of
\emph{nodes} that represent some objects (in this case the \emph{classes} 
of our \emph{model}).The instruction \emph{applyLayout} sets the method 
the \emph{cubes} will follow to display themselves (in this case on a 
sphere). And after adding another \emph{cube} at the center (the basic 
position for any \emph{cube}), by using \emph{edges}, it defines the way 
every \emph{edge} will be draw (the \emph{edge} depends on the previous 
definition of the \emph{cubes}).

\begin{lstlisting}
| elementsToVisualize |
elementsToVisualize := KLEaselCommand  subclasses.
view nodes: elementsToVisualize.
view applyLayout: KLSphereLayout new.

view node: KLEaselCommand using: (KLCube new fillColor: Color random).
view edges: elementsToVisualize from: #yourself to: #superclass.   
\end{lstlisting}

\fig{}{0.5}{figure1.png}{KLEaselCommand subclasses.}


%=======
\subsection{Scripting visualizations with Klotz} \seclabel{scripting}
The Klotz scripting l\chg{e}{a}nguage is based on Pharo Smalltalk
(since the application is made on it) and 4 basic principles
to manage the interaction between the elements on the \emph{view}:
The use of \emph{nodes}(i) to represent the elements that are \del{been}
displayed, \emph{edges}(ii) to represent the relationship between
the selected \emph{nodes}, \emph{Layouts}(iii) to define the way the
\emph{view} distributes the objetcs on itself, and a special value
system defined by Pharo symbols used to access to properties of
the object being modeled (iv).  

The scripting language supports 4 diferent ways of defining \emph{nodes}:
The instructions \emph{node:} and \emph{nodes:} define standard 
\emph{nodes}(blue colored, "1" length edged \emph{cubes}), to set 1 or many 
\emph{nodes} at the same time, respectively.
\begin{lstlisting}
	view node: Magnitude.
	view node: Number.
	view node: Time    
\end{lstlisting}
\fig{}{0.3}{figure2.png}{Magnitude class with 2 of his subclasses.}
Other way to define them, and customise the \emph{cubes} related to the 
\emph{nodes}, is by using the instructions \emph{node: using:} and 
\emph{nodes: using:}, both instructions are meant to directly define 
the \emph{cube} that will represent the inner \emph{model} of the \emph{node}, 
or each \emph{node} in the second case.
\begin{lstlisting}
view nodes: Magnitude subclasses using: (KLCube new height: #numberOfInstanceVariables)    
\end{lstlisting}
\fig{}{0.5}{figure3.png}{Magnitude subclasses, the heights shows the number of instance variables.}

Just like it happens with the \emph{nodes}, the \emph{edges} can be defined in 
4 ways: The instructions \emph{edge: from: to:} and \emph{edges: from: to:} 
apply, respectively, the \emph{edge} (\emph{edges}) between 2 cubes, each \emph{edge}.
\begin{lstlisting}
view nodes: (1 to: 5).	
view edges: (1 to: 4) from: #yourself to: 5 
\end{lstlisting}
\fig{}{0.5}{figure4.png}{Basic edges}

And also as the previous definition the keyword \emph{using:} is used
to define the metrics of the \emph{edge} itself.  
\begin{lstlisting}
view nodes: (1 to: 5).	
view edges: (1 to: 4) from: #yourself to: 5 using: (KLLine new width: #yourself)
\end{lstlisting}
\fig{}{0.5}{figure5.png}{Basic edges, using metrics}

To apply \emph{Layouts} on the current \emph{nodes} the scripting
language uses the instruction \emph{applyLayout:}. \emph{Layouts}
define the methods (or metrics) that will be used to decide how
to display the elements.

The application of a \emph{Layout} will affect every \emph{node} 
(\emph{edge}) defined before itself.
\begin{lstlisting}
| colors |
colors := (Color colorNames collect: 
	 [:each | Color fromString: (each asString)]). "colors as Colors"
view nodes: colors using: (KLCube new fillColor: #yourself).
view applyLayout: (KLCubeLayout new).
\end{lstlisting}
\fig{}{0.3}{figure6.png}{All Colors avaible in Pharo}
 
%=======
\subsection{The third dimension} \seclabel{3d}

\paragraph{Distinguish the depth} 
A visualization contains one unique light. 
A colorless light is at the same position than the camera. 
Maximum intensity when a face has orthogonal to the camera.

Perspective? 


\paragraph{Commands in the easel} 
The easel contains XX commands to rotate 

\paragraph{Layout}

\paragraph{Composition}

%=======
\subsection{Properties} \seclabel{properties}

\begin{itemize}
\item \emph{The visualization engine should be domain independent.}
\item \emph{Visualizations should be easily composed from simpler parts.}
\item \emph{The visualization should be definable at a fine grained level.}
\item \emph{Object creation overhead should be kept to a minimum.}
\item \emph{The visualization description should be declarative.}
\end{itemize}

%:%%%%%%%%%%%%%%%%%%%%%%%%%%%%%%%%%%%%%%%%%%%%%%%%%%
\section{Applications} \seclabel{applications}

%=======
\subsection{Codecity}

%=======
\subsection{Scatterplot}


%:%%%%%%%%%%%%%%%%%%%%%%%%%%%%%%%%%%%%%%%%%%%%%%%%%%
\section{Implementation} \seclabel{implementation}

%:%%%%%%%%%%%%%%%%%%%%%%%%%%%%%%%%%%%%%%%%%%%%%%%%%%
\section{Related Work} \seclabel{relatedwork}

Lumiere~\cite{Oliv09a}

CodeCity~\cite{Wett08d}

%:%%%%%%%%%%%%%%%%%%%%%%%%%%%%%%%%%%%%%%%%%%%%%%%%%%
\section{Conclusion} \seclabel{conclusion}



%{\small 
%\paragraph{Acknowledgment.} We thank Mircea Lungu, Oscar Nierstrasz, Lukas Renggli and Romain Robbes for the multiple discussions we had and their comments on an early draft of the paper.
%We particular thank Walter Binder for his multiple discussions and advices.
%Our thanks also go to Eliot Miranda for his help on porting \compteur to Cog, the jitted virtual machine of Pharo. 
%We thank Gilad Bracha and Jan Vran\'{y} for the fruitful discussions we had.
%We also thank Andrew P. Black for his precious help on improving the paper.
%We gratefully thank Mar\'ia Jos\'e Cires for her help on the statistical part. 
%We also thank ESUG, the European Smalltalk User Group, for its financial contribution to the presentation of this paper. 
%}
%

\bibliographystyle{plain}
\bibliography{scg}


\end{document}
